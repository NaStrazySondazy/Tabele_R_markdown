\documentclass[a4paper, 10pt]{article}\usepackage[]{graphicx}\usepackage[]{color}
%% maxwidth is the original width if it is less than linewidth
%% otherwise use linewidth (to make sure the graphics do not exceed the margin)
\makeatletter
\def\maxwidth{ %
  \ifdim\Gin@nat@width>\linewidth
    \linewidth
  \else
    \Gin@nat@width
  \fi
}
\makeatother

\definecolor{fgcolor}{rgb}{0.345, 0.345, 0.345}
\newcommand{\hlnum}[1]{\textcolor[rgb]{0.686,0.059,0.569}{#1}}%
\newcommand{\hlstr}[1]{\textcolor[rgb]{0.192,0.494,0.8}{#1}}%
\newcommand{\hlcom}[1]{\textcolor[rgb]{0.678,0.584,0.686}{\textit{#1}}}%
\newcommand{\hlopt}[1]{\textcolor[rgb]{0,0,0}{#1}}%
\newcommand{\hlstd}[1]{\textcolor[rgb]{0.345,0.345,0.345}{#1}}%
\newcommand{\hlkwa}[1]{\textcolor[rgb]{0.161,0.373,0.58}{\textbf{#1}}}%
\newcommand{\hlkwb}[1]{\textcolor[rgb]{0.69,0.353,0.396}{#1}}%
\newcommand{\hlkwc}[1]{\textcolor[rgb]{0.333,0.667,0.333}{#1}}%
\newcommand{\hlkwd}[1]{\textcolor[rgb]{0.737,0.353,0.396}{\textbf{#1}}}%

\usepackage{framed}
\makeatletter
\newenvironment{kframe}{%
 \def\at@end@of@kframe{}%
 \ifinner\ifhmode%
  \def\at@end@of@kframe{\end{minipage}}%
  \begin{minipage}{\columnwidth}%
 \fi\fi%
 \def\FrameCommand##1{\hskip\@totalleftmargin \hskip-\fboxsep
 \colorbox{shadecolor}{##1}\hskip-\fboxsep
     % There is no \\@totalrightmargin, so:
     \hskip-\linewidth \hskip-\@totalleftmargin \hskip\columnwidth}%
 \MakeFramed {\advance\hsize-\width
   \@totalleftmargin\z@ \linewidth\hsize
   \@setminipage}}%
 {\par\unskip\endMakeFramed%
 \at@end@of@kframe}
\makeatother

\definecolor{shadecolor}{rgb}{.97, .97, .97}
\definecolor{messagecolor}{rgb}{0, 0, 0}
\definecolor{warningcolor}{rgb}{1, 0, 1}
\definecolor{errorcolor}{rgb}{1, 0, 0}
\newenvironment{knitrout}{}{} % an empty environment to be redefined in TeX

\usepackage{alltt} % ustawienia dokumentu
\usepackage{polski} % język
\usepackage[utf8]{inputenc} % język
\setlength{\parindent}{0pt} % wciecie akapitu
\usepackage[margin=1in]{geometry} % marginesy
\linespread{1.3} % interlinia 1,5 
\usepackage{longtable} % ustawienia dla tabel - bliblioteka longtable
\usepackage{booktabs} % ustawienia dla tabel - bliblioteka booktabs
\IfFileExists{upquote.sty}{\usepackage{upquote}}{}
\begin{document}

\title{Generowanie tabele w LaTeX-u (\emph{Sweave}) przy użyciu funkcji \emph{kable}}
\date{9 czerwca 2015 r.}
\author{ Zbyszek Marczewski}

\maketitle

\tableofcontents

\section{Wstęp}

Oto krótki opis moich doświadczeń z tabelami w dokumentach PDF kompilowanych za pomocą \emph{RStudio} na podstawie plików \emph{.Rmw}.
\vspace{5mm}

Poniższy dokument powstał przy użyciu narzędzia \emph{Sweave} stanowiącego integralną część R. \emph{Sweave} jest produktem niemieckiej myśli technicznej z początku XXI wieku (z tego co wiem jego autorem jest Friedrich Leisch – obecnie Institut für Statistik der Ludwig-Maximilians-Universität München) . Pierwotnym zastosowaniem tego pakietu było tworzenie artykułów naukowych, w których prezentowane są obliczenia przeprowadzone w R. 
\vspace{5mm}

\emph{Sweave} działa podobnie jak \emph{Markdown}. Podstawowa różnicą polega na tym, że cały dokument  piszemy w LaTeX-u, a plik ma rozszeczenie \emph{.Rmw}. Natomiast Obliczenia w R wrzuca się do \emph{Sweave} tak samo jak w \emph{Markdownie}, czyli w postaci bloków kodu, które różnią się tylko znakami otwarcia i zamknięcia:

\begin{itemize}
  \item \emph{<< >>=} - otwarcie
  \item \emph{@} - zamknięcie
\end{itemize}

Same obliczenia i ich sposób wyświetlania jest identyczny jak w \emph{Markdown}.

\begin{knitrout}
\definecolor{shadecolor}{rgb}{0.969, 0.969, 0.969}\color{fgcolor}\begin{kframe}
\begin{alltt}
\hlnum{2}\hlopt{+}\hlnum{2}
\end{alltt}
\begin{verbatim}
## [1] 4
\end{verbatim}
\end{kframe}
\end{knitrout}

Aby użyć \emph{Sweave} wystarczy wybrać w \emph{RStudio} \textbf{File $\rightarrow$ New File $\rightarrow$ R Sweave}. Otworzy nam się wtedy nowy plik \emph{.Rnw}, w który używamy LaTeX-a i który kompilujemy do PDF przy użyciu przycisku "Compile PDF". Zupełnie jak w przypadku \emph{Markdown} :-)
\vspace{5mm}

Poniżej znajduje się kilka tabel wygenerowanych w \emph{Sweave} przy użyciu różnych metod. Tabele zawierają po 80 pierwszych wierszy ze zbioru \textbf{iris}. UWAGA!!!  Aby wyświetlić tabelę generowaną z R przy pomocy \emph{Sweave}, podobnie jak w \emph{Markdown}, musimy w opcjach bloku kodu (\emph{chunk options}) użyć argumentu \textbf{results='asis'}. 

\section{Klasyczne rozwiązanie - ``xtable''}

Klasycznym rozwiązaniem stosowanym w celu wstawianienia tabeli z R do pliku PDF jest użycie funkcji \textbf{xtable} z pakietu \emph{xtable}. W ten sposób możemy kontrolować wiele cech tabeli. Niestety jest to rozwiązanie czasochłonne - konieczność ustawiania wielu parametrów.

\begin{kframe}
\begin{alltt}
\hlstd{tabela.2.1}\hlkwb{<-}\hlstd{xtable}\hlopt{::} \hlkwd{xtable}\hlstd{(}\hlkwd{head}\hlstd{(iris,} \hlnum{80}\hlstd{),}
                            \hlkwc{caption}\hlstd{=}\hlstr{"Zastosowanie funkcji xtable "}\hlstd{)}
\hlkwd{print}\hlstd{( tabela.2.1,} \hlcom{# tabela}
       \hlkwc{table.placement} \hlstd{=} \hlstr{"H"}\hlstd{,} \hlcom{# położenie tabeli - }
       \hlkwc{caption.placement} \hlstd{=} \hlstr{"top"}\hlstd{,} \hlcom{# położenie tytułu}
       \hlkwc{include.rownames} \hlstd{=} \hlnum{TRUE}\hlstd{,} \hlcom{# uwzględnienie nazw wierszy}
       \hlkwc{include.colnames} \hlstd{=} \hlnum{TRUE}\hlstd{,} \hlcom{# uwzględnienie nazw kolum}
       \hlkwc{size} \hlstd{=} \hlstr{"small"}\hlstd{,} \hlcom{# rozmiar tebeli}
       \hlkwc{tabular.environment} \hlstd{=} \hlstr{'longtable'}\hlstd{,} \hlcom{# tabela na kilku stronach}
       \hlkwc{floating} \hlstd{=} \hlnum{FALSE}\hlstd{,} \hlcom{# "pływanie" tabeli po stronie}
       \hlcom{# powtarzanie pierwszego wiersza na każdej stronie}
       \hlkwc{add.to.row} \hlstd{=} \hlkwd{list}\hlstd{(}\hlkwc{pos} \hlstd{=} \hlkwd{list}\hlstd{(}\hlnum{0}\hlstd{),} \hlkwc{command} \hlstd{=} \hlstr{"\textbackslash{}\textbackslash{}hline \textbackslash{}\textbackslash{}endhead "}\hlstd{)}
\hlstd{)}
\end{alltt}
\end{kframe}% latex table generated in R 3.1.1 by xtable 1.7-4 package
% Wed Jun 10 10:12:48 2015
{\small
\begin{longtable}{rrrrrl}
\caption{Zastosowanie funkcji xtable } \\ 
  \hline
 & Sepal.Length & Sepal.Width & Petal.Length & Petal.Width & Species \\ 
  \hline \endhead  \hline
1 & 5.10 & 3.50 & 1.40 & 0.20 & setosa \\ 
  2 & 4.90 & 3.00 & 1.40 & 0.20 & setosa \\ 
  3 & 4.70 & 3.20 & 1.30 & 0.20 & setosa \\ 
  4 & 4.60 & 3.10 & 1.50 & 0.20 & setosa \\ 
  5 & 5.00 & 3.60 & 1.40 & 0.20 & setosa \\ 
  6 & 5.40 & 3.90 & 1.70 & 0.40 & setosa \\ 
  7 & 4.60 & 3.40 & 1.40 & 0.30 & setosa \\ 
  8 & 5.00 & 3.40 & 1.50 & 0.20 & setosa \\ 
  9 & 4.40 & 2.90 & 1.40 & 0.20 & setosa \\ 
  10 & 4.90 & 3.10 & 1.50 & 0.10 & setosa \\ 
  11 & 5.40 & 3.70 & 1.50 & 0.20 & setosa \\ 
  12 & 4.80 & 3.40 & 1.60 & 0.20 & setosa \\ 
  13 & 4.80 & 3.00 & 1.40 & 0.10 & setosa \\ 
  14 & 4.30 & 3.00 & 1.10 & 0.10 & setosa \\ 
  15 & 5.80 & 4.00 & 1.20 & 0.20 & setosa \\ 
  16 & 5.70 & 4.40 & 1.50 & 0.40 & setosa \\ 
  17 & 5.40 & 3.90 & 1.30 & 0.40 & setosa \\ 
  18 & 5.10 & 3.50 & 1.40 & 0.30 & setosa \\ 
  19 & 5.70 & 3.80 & 1.70 & 0.30 & setosa \\ 
  20 & 5.10 & 3.80 & 1.50 & 0.30 & setosa \\ 
  21 & 5.40 & 3.40 & 1.70 & 0.20 & setosa \\ 
  22 & 5.10 & 3.70 & 1.50 & 0.40 & setosa \\ 
  23 & 4.60 & 3.60 & 1.00 & 0.20 & setosa \\ 
  24 & 5.10 & 3.30 & 1.70 & 0.50 & setosa \\ 
  25 & 4.80 & 3.40 & 1.90 & 0.20 & setosa \\ 
  26 & 5.00 & 3.00 & 1.60 & 0.20 & setosa \\ 
  27 & 5.00 & 3.40 & 1.60 & 0.40 & setosa \\ 
  28 & 5.20 & 3.50 & 1.50 & 0.20 & setosa \\ 
  29 & 5.20 & 3.40 & 1.40 & 0.20 & setosa \\ 
  30 & 4.70 & 3.20 & 1.60 & 0.20 & setosa \\ 
  31 & 4.80 & 3.10 & 1.60 & 0.20 & setosa \\ 
  32 & 5.40 & 3.40 & 1.50 & 0.40 & setosa \\ 
  33 & 5.20 & 4.10 & 1.50 & 0.10 & setosa \\ 
  34 & 5.50 & 4.20 & 1.40 & 0.20 & setosa \\ 
  35 & 4.90 & 3.10 & 1.50 & 0.20 & setosa \\ 
  36 & 5.00 & 3.20 & 1.20 & 0.20 & setosa \\ 
  37 & 5.50 & 3.50 & 1.30 & 0.20 & setosa \\ 
  38 & 4.90 & 3.60 & 1.40 & 0.10 & setosa \\ 
  39 & 4.40 & 3.00 & 1.30 & 0.20 & setosa \\ 
  40 & 5.10 & 3.40 & 1.50 & 0.20 & setosa \\ 
  41 & 5.00 & 3.50 & 1.30 & 0.30 & setosa \\ 
  42 & 4.50 & 2.30 & 1.30 & 0.30 & setosa \\ 
  43 & 4.40 & 3.20 & 1.30 & 0.20 & setosa \\ 
  44 & 5.00 & 3.50 & 1.60 & 0.60 & setosa \\ 
  45 & 5.10 & 3.80 & 1.90 & 0.40 & setosa \\ 
  46 & 4.80 & 3.00 & 1.40 & 0.30 & setosa \\ 
  47 & 5.10 & 3.80 & 1.60 & 0.20 & setosa \\ 
  48 & 4.60 & 3.20 & 1.40 & 0.20 & setosa \\ 
  49 & 5.30 & 3.70 & 1.50 & 0.20 & setosa \\ 
  50 & 5.00 & 3.30 & 1.40 & 0.20 & setosa \\ 
  51 & 7.00 & 3.20 & 4.70 & 1.40 & versicolor \\ 
  52 & 6.40 & 3.20 & 4.50 & 1.50 & versicolor \\ 
  53 & 6.90 & 3.10 & 4.90 & 1.50 & versicolor \\ 
  54 & 5.50 & 2.30 & 4.00 & 1.30 & versicolor \\ 
  55 & 6.50 & 2.80 & 4.60 & 1.50 & versicolor \\ 
  56 & 5.70 & 2.80 & 4.50 & 1.30 & versicolor \\ 
  57 & 6.30 & 3.30 & 4.70 & 1.60 & versicolor \\ 
  58 & 4.90 & 2.40 & 3.30 & 1.00 & versicolor \\ 
  59 & 6.60 & 2.90 & 4.60 & 1.30 & versicolor \\ 
  60 & 5.20 & 2.70 & 3.90 & 1.40 & versicolor \\ 
  61 & 5.00 & 2.00 & 3.50 & 1.00 & versicolor \\ 
  62 & 5.90 & 3.00 & 4.20 & 1.50 & versicolor \\ 
  63 & 6.00 & 2.20 & 4.00 & 1.00 & versicolor \\ 
  64 & 6.10 & 2.90 & 4.70 & 1.40 & versicolor \\ 
  65 & 5.60 & 2.90 & 3.60 & 1.30 & versicolor \\ 
  66 & 6.70 & 3.10 & 4.40 & 1.40 & versicolor \\ 
  67 & 5.60 & 3.00 & 4.50 & 1.50 & versicolor \\ 
  68 & 5.80 & 2.70 & 4.10 & 1.00 & versicolor \\ 
  69 & 6.20 & 2.20 & 4.50 & 1.50 & versicolor \\ 
  70 & 5.60 & 2.50 & 3.90 & 1.10 & versicolor \\ 
  71 & 5.90 & 3.20 & 4.80 & 1.80 & versicolor \\ 
  72 & 6.10 & 2.80 & 4.00 & 1.30 & versicolor \\ 
  73 & 6.30 & 2.50 & 4.90 & 1.50 & versicolor \\ 
  74 & 6.10 & 2.80 & 4.70 & 1.20 & versicolor \\ 
  75 & 6.40 & 2.90 & 4.30 & 1.30 & versicolor \\ 
  76 & 6.60 & 3.00 & 4.40 & 1.40 & versicolor \\ 
  77 & 6.80 & 2.80 & 4.80 & 1.40 & versicolor \\ 
  78 & 6.70 & 3.00 & 5.00 & 1.70 & versicolor \\ 
  79 & 6.00 & 2.90 & 4.50 & 1.50 & versicolor \\ 
  80 & 5.70 & 2.60 & 3.50 & 1.00 & versicolor \\ 
   \hline
\hline
\end{longtable}
}


\section{Prostsze rozwiązanie - ``kable''}

Innym rozwiązaniem, chyba mniej znanym, jest zastosowanie funkcji \textbf{kable} z pakietu \emph{knitr}. Działa ona zarówno w dokumentach rmarkdownowych, przy kompilowaniu plików HTML, PDF i MsWord (patrz: Tabele\_rmarkdown.pdf, Tabele\_rmarkdown.html, Tabele\_rmarkdown.docx), jak i przy produkowaniu dokumnetów za pomocą \emph{Sweave} (czyli w LaTeX-u). Podstawową jej zaletą jest prostota. Wszystko ogranicza się do kilku argumentów, co jedni uznają za zaletę - mniej roboty, a inni za wadę - brak pełnej kontroli. 

\subsection{Zwykłe ``kable''}

Poniżej, przy użyciu funkcji \textbf{kable}, stworzono najprostrzą z możliwych tabel. Jak widać nie wszystko się udało. Tabela nie wygląda ładnie, a na dodatek nie przechodzi na kolejną stronę.

\begin{kframe}
\begin{alltt}
\hlstd{knitr}\hlopt{::} \hlkwd{kable}\hlstd{(} \hlkwd{head}\hlstd{(iris,} \hlnum{80}\hlstd{) ,} \hlkwc{format} \hlstd{=}\hlstr{"latex"}\hlstd{)}
\end{alltt}
\end{kframe}
\begin{tabular}{r|r|r|r|l}
\hline
Sepal.Length & Sepal.Width & Petal.Length & Petal.Width & Species\\
\hline
5.1 & 3.5 & 1.4 & 0.2 & setosa\\
\hline
4.9 & 3.0 & 1.4 & 0.2 & setosa\\
\hline
4.7 & 3.2 & 1.3 & 0.2 & setosa\\
\hline
4.6 & 3.1 & 1.5 & 0.2 & setosa\\
\hline
5.0 & 3.6 & 1.4 & 0.2 & setosa\\
\hline
5.4 & 3.9 & 1.7 & 0.4 & setosa\\
\hline
4.6 & 3.4 & 1.4 & 0.3 & setosa\\
\hline
5.0 & 3.4 & 1.5 & 0.2 & setosa\\
\hline
4.4 & 2.9 & 1.4 & 0.2 & setosa\\
\hline
4.9 & 3.1 & 1.5 & 0.1 & setosa\\
\hline
5.4 & 3.7 & 1.5 & 0.2 & setosa\\
\hline
4.8 & 3.4 & 1.6 & 0.2 & setosa\\
\hline
4.8 & 3.0 & 1.4 & 0.1 & setosa\\
\hline
4.3 & 3.0 & 1.1 & 0.1 & setosa\\
\hline
5.8 & 4.0 & 1.2 & 0.2 & setosa\\
\hline
5.7 & 4.4 & 1.5 & 0.4 & setosa\\
\hline
5.4 & 3.9 & 1.3 & 0.4 & setosa\\
\hline
5.1 & 3.5 & 1.4 & 0.3 & setosa\\
\hline
5.7 & 3.8 & 1.7 & 0.3 & setosa\\
\hline
5.1 & 3.8 & 1.5 & 0.3 & setosa\\
\hline
5.4 & 3.4 & 1.7 & 0.2 & setosa\\
\hline
5.1 & 3.7 & 1.5 & 0.4 & setosa\\
\hline
4.6 & 3.6 & 1.0 & 0.2 & setosa\\
\hline
5.1 & 3.3 & 1.7 & 0.5 & setosa\\
\hline
4.8 & 3.4 & 1.9 & 0.2 & setosa\\
\hline
5.0 & 3.0 & 1.6 & 0.2 & setosa\\
\hline
5.0 & 3.4 & 1.6 & 0.4 & setosa\\
\hline
5.2 & 3.5 & 1.5 & 0.2 & setosa\\
\hline
5.2 & 3.4 & 1.4 & 0.2 & setosa\\
\hline
4.7 & 3.2 & 1.6 & 0.2 & setosa\\
\hline
4.8 & 3.1 & 1.6 & 0.2 & setosa\\
\hline
5.4 & 3.4 & 1.5 & 0.4 & setosa\\
\hline
5.2 & 4.1 & 1.5 & 0.1 & setosa\\
\hline
5.5 & 4.2 & 1.4 & 0.2 & setosa\\
\hline
4.9 & 3.1 & 1.5 & 0.2 & setosa\\
\hline
5.0 & 3.2 & 1.2 & 0.2 & setosa\\
\hline
5.5 & 3.5 & 1.3 & 0.2 & setosa\\
\hline
4.9 & 3.6 & 1.4 & 0.1 & setosa\\
\hline
4.4 & 3.0 & 1.3 & 0.2 & setosa\\
\hline
5.1 & 3.4 & 1.5 & 0.2 & setosa\\
\hline
5.0 & 3.5 & 1.3 & 0.3 & setosa\\
\hline
4.5 & 2.3 & 1.3 & 0.3 & setosa\\
\hline
4.4 & 3.2 & 1.3 & 0.2 & setosa\\
\hline
5.0 & 3.5 & 1.6 & 0.6 & setosa\\
\hline
5.1 & 3.8 & 1.9 & 0.4 & setosa\\
\hline
4.8 & 3.0 & 1.4 & 0.3 & setosa\\
\hline
5.1 & 3.8 & 1.6 & 0.2 & setosa\\
\hline
4.6 & 3.2 & 1.4 & 0.2 & setosa\\
\hline
5.3 & 3.7 & 1.5 & 0.2 & setosa\\
\hline
5.0 & 3.3 & 1.4 & 0.2 & setosa\\
\hline
7.0 & 3.2 & 4.7 & 1.4 & versicolor\\
\hline
6.4 & 3.2 & 4.5 & 1.5 & versicolor\\
\hline
6.9 & 3.1 & 4.9 & 1.5 & versicolor\\
\hline
5.5 & 2.3 & 4.0 & 1.3 & versicolor\\
\hline
6.5 & 2.8 & 4.6 & 1.5 & versicolor\\
\hline
5.7 & 2.8 & 4.5 & 1.3 & versicolor\\
\hline
6.3 & 3.3 & 4.7 & 1.6 & versicolor\\
\hline
4.9 & 2.4 & 3.3 & 1.0 & versicolor\\
\hline
6.6 & 2.9 & 4.6 & 1.3 & versicolor\\
\hline
5.2 & 2.7 & 3.9 & 1.4 & versicolor\\
\hline
5.0 & 2.0 & 3.5 & 1.0 & versicolor\\
\hline
5.9 & 3.0 & 4.2 & 1.5 & versicolor\\
\hline
6.0 & 2.2 & 4.0 & 1.0 & versicolor\\
\hline
6.1 & 2.9 & 4.7 & 1.4 & versicolor\\
\hline
5.6 & 2.9 & 3.6 & 1.3 & versicolor\\
\hline
6.7 & 3.1 & 4.4 & 1.4 & versicolor\\
\hline
5.6 & 3.0 & 4.5 & 1.5 & versicolor\\
\hline
5.8 & 2.7 & 4.1 & 1.0 & versicolor\\
\hline
6.2 & 2.2 & 4.5 & 1.5 & versicolor\\
\hline
5.6 & 2.5 & 3.9 & 1.1 & versicolor\\
\hline
5.9 & 3.2 & 4.8 & 1.8 & versicolor\\
\hline
6.1 & 2.8 & 4.0 & 1.3 & versicolor\\
\hline
6.3 & 2.5 & 4.9 & 1.5 & versicolor\\
\hline
6.1 & 2.8 & 4.7 & 1.2 & versicolor\\
\hline
6.4 & 2.9 & 4.3 & 1.3 & versicolor\\
\hline
6.6 & 3.0 & 4.4 & 1.4 & versicolor\\
\hline
6.8 & 2.8 & 4.8 & 1.4 & versicolor\\
\hline
6.7 & 3.0 & 5.0 & 1.7 & versicolor\\
\hline
6.0 & 2.9 & 4.5 & 1.5 & versicolor\\
\hline
5.7 & 2.6 & 3.5 & 1.0 & versicolor\\
\hline
\end{tabular}



\subsection{Lepsze ``kable'' z użyciem biblioteki \emph{longtable}}

Aby uzyskać przy użyciu funkcji \textbf{kable} tabelę, która przejdzie na następne strony, wystarczy dodać jeden argument \textbf{longtable=TRUE} oraz uruchomić w LaTeX-u, na początku pliku \emph{.Rnw}, bibliotekę \textbf{longtable} (patrz 7 wiersz pliku Tabele\_LATEX.Rnw). Dzięku temu tabela zostanie przeniesiona na następne strony.

\begin{kframe}
\begin{alltt}
\hlstd{knitr}\hlopt{::} \hlkwd{kable}\hlstd{(} \hlkwd{head}\hlstd{(iris,} \hlnum{80}\hlstd{),} \hlkwc{longtable}\hlstd{=}\hlnum{TRUE}\hlstd{)}
\end{alltt}
\end{kframe}
\begin{longtable}{r|r|r|r|l}
\hline
Sepal.Length & Sepal.Width & Petal.Length & Petal.Width & Species\\
\hline
5.1 & 3.5 & 1.4 & 0.2 & setosa\\
\hline
4.9 & 3.0 & 1.4 & 0.2 & setosa\\
\hline
4.7 & 3.2 & 1.3 & 0.2 & setosa\\
\hline
4.6 & 3.1 & 1.5 & 0.2 & setosa\\
\hline
5.0 & 3.6 & 1.4 & 0.2 & setosa\\
\hline
5.4 & 3.9 & 1.7 & 0.4 & setosa\\
\hline
4.6 & 3.4 & 1.4 & 0.3 & setosa\\
\hline
5.0 & 3.4 & 1.5 & 0.2 & setosa\\
\hline
4.4 & 2.9 & 1.4 & 0.2 & setosa\\
\hline
4.9 & 3.1 & 1.5 & 0.1 & setosa\\
\hline
5.4 & 3.7 & 1.5 & 0.2 & setosa\\
\hline
4.8 & 3.4 & 1.6 & 0.2 & setosa\\
\hline
4.8 & 3.0 & 1.4 & 0.1 & setosa\\
\hline
4.3 & 3.0 & 1.1 & 0.1 & setosa\\
\hline
5.8 & 4.0 & 1.2 & 0.2 & setosa\\
\hline
5.7 & 4.4 & 1.5 & 0.4 & setosa\\
\hline
5.4 & 3.9 & 1.3 & 0.4 & setosa\\
\hline
5.1 & 3.5 & 1.4 & 0.3 & setosa\\
\hline
5.7 & 3.8 & 1.7 & 0.3 & setosa\\
\hline
5.1 & 3.8 & 1.5 & 0.3 & setosa\\
\hline
5.4 & 3.4 & 1.7 & 0.2 & setosa\\
\hline
5.1 & 3.7 & 1.5 & 0.4 & setosa\\
\hline
4.6 & 3.6 & 1.0 & 0.2 & setosa\\
\hline
5.1 & 3.3 & 1.7 & 0.5 & setosa\\
\hline
4.8 & 3.4 & 1.9 & 0.2 & setosa\\
\hline
5.0 & 3.0 & 1.6 & 0.2 & setosa\\
\hline
5.0 & 3.4 & 1.6 & 0.4 & setosa\\
\hline
5.2 & 3.5 & 1.5 & 0.2 & setosa\\
\hline
5.2 & 3.4 & 1.4 & 0.2 & setosa\\
\hline
4.7 & 3.2 & 1.6 & 0.2 & setosa\\
\hline
4.8 & 3.1 & 1.6 & 0.2 & setosa\\
\hline
5.4 & 3.4 & 1.5 & 0.4 & setosa\\
\hline
5.2 & 4.1 & 1.5 & 0.1 & setosa\\
\hline
5.5 & 4.2 & 1.4 & 0.2 & setosa\\
\hline
4.9 & 3.1 & 1.5 & 0.2 & setosa\\
\hline
5.0 & 3.2 & 1.2 & 0.2 & setosa\\
\hline
5.5 & 3.5 & 1.3 & 0.2 & setosa\\
\hline
4.9 & 3.6 & 1.4 & 0.1 & setosa\\
\hline
4.4 & 3.0 & 1.3 & 0.2 & setosa\\
\hline
5.1 & 3.4 & 1.5 & 0.2 & setosa\\
\hline
5.0 & 3.5 & 1.3 & 0.3 & setosa\\
\hline
4.5 & 2.3 & 1.3 & 0.3 & setosa\\
\hline
4.4 & 3.2 & 1.3 & 0.2 & setosa\\
\hline
5.0 & 3.5 & 1.6 & 0.6 & setosa\\
\hline
5.1 & 3.8 & 1.9 & 0.4 & setosa\\
\hline
4.8 & 3.0 & 1.4 & 0.3 & setosa\\
\hline
5.1 & 3.8 & 1.6 & 0.2 & setosa\\
\hline
4.6 & 3.2 & 1.4 & 0.2 & setosa\\
\hline
5.3 & 3.7 & 1.5 & 0.2 & setosa\\
\hline
5.0 & 3.3 & 1.4 & 0.2 & setosa\\
\hline
7.0 & 3.2 & 4.7 & 1.4 & versicolor\\
\hline
6.4 & 3.2 & 4.5 & 1.5 & versicolor\\
\hline
6.9 & 3.1 & 4.9 & 1.5 & versicolor\\
\hline
5.5 & 2.3 & 4.0 & 1.3 & versicolor\\
\hline
6.5 & 2.8 & 4.6 & 1.5 & versicolor\\
\hline
5.7 & 2.8 & 4.5 & 1.3 & versicolor\\
\hline
6.3 & 3.3 & 4.7 & 1.6 & versicolor\\
\hline
4.9 & 2.4 & 3.3 & 1.0 & versicolor\\
\hline
6.6 & 2.9 & 4.6 & 1.3 & versicolor\\
\hline
5.2 & 2.7 & 3.9 & 1.4 & versicolor\\
\hline
5.0 & 2.0 & 3.5 & 1.0 & versicolor\\
\hline
5.9 & 3.0 & 4.2 & 1.5 & versicolor\\
\hline
6.0 & 2.2 & 4.0 & 1.0 & versicolor\\
\hline
6.1 & 2.9 & 4.7 & 1.4 & versicolor\\
\hline
5.6 & 2.9 & 3.6 & 1.3 & versicolor\\
\hline
6.7 & 3.1 & 4.4 & 1.4 & versicolor\\
\hline
5.6 & 3.0 & 4.5 & 1.5 & versicolor\\
\hline
5.8 & 2.7 & 4.1 & 1.0 & versicolor\\
\hline
6.2 & 2.2 & 4.5 & 1.5 & versicolor\\
\hline
5.6 & 2.5 & 3.9 & 1.1 & versicolor\\
\hline
5.9 & 3.2 & 4.8 & 1.8 & versicolor\\
\hline
6.1 & 2.8 & 4.0 & 1.3 & versicolor\\
\hline
6.3 & 2.5 & 4.9 & 1.5 & versicolor\\
\hline
6.1 & 2.8 & 4.7 & 1.2 & versicolor\\
\hline
6.4 & 2.9 & 4.3 & 1.3 & versicolor\\
\hline
6.6 & 3.0 & 4.4 & 1.4 & versicolor\\
\hline
6.8 & 2.8 & 4.8 & 1.4 & versicolor\\
\hline
6.7 & 3.0 & 5.0 & 1.7 & versicolor\\
\hline
6.0 & 2.9 & 4.5 & 1.5 & versicolor\\
\hline
5.7 & 2.6 & 3.5 & 1.0 & versicolor\\
\hline
\end{longtable}



\subsection{Najlesze ``kable'' z użyciem bibliotek \emph{longtable} i \emph{booktabs}}

Ale to nie wszystko. Dodatkowo do funkcji \textbf{kable} można dołożyć bibliotekę \textbf{booktabs}, która nada tabeli ładniejszy wygląd. Poniżej zastosowanie funkcji \textbf{kable} z jej wszystkimi możliwościami, w tym z dodaniem tytyłu.


\begin{kframe}
\begin{alltt}
\hlstd{knitr}\hlopt{::} \hlkwd{kable}\hlstd{(} \hlkwd{head}\hlstd{(iris,} \hlnum{80}\hlstd{) ,}
        \hlkwc{caption}\hlstd{=}\hlstr{"Tabela 4. Wykorzystanie wyszystkich możliwości funkcji kable + tytuł"}\hlstd{,}
        \hlkwc{row.names} \hlstd{= T,} \hlcom{# nazwy wierszy}
        \hlkwc{format}\hlstd{=}\hlstr{"latex"}\hlstd{,} \hlcom{# format LaTeX}
        \hlkwc{longtable}\hlstd{=}\hlnum{TRUE}\hlstd{,}  \hlcom{# biblioteka longtable}
        \hlkwc{booktabs}\hlstd{=}\hlnum{TRUE}\hlstd{,} \hlcom{# biblioteka booktabs}
        \hlkwc{align}\hlstd{=}\hlkwd{c}\hlstd{(}\hlstr{'l'}\hlstd{,} \hlstr{'c'}\hlstd{,} \hlstr{'r'}\hlstd{,} \hlstr{'l'}\hlstd{,} \hlstr{'r'}\hlstd{),} \hlcom{# wyrównanie w kolumanch}
        \hlkwc{digits}\hlstd{=}\hlkwd{c}\hlstd{(}\hlnum{0}\hlstd{,}\hlnum{1}\hlstd{,}\hlnum{2}\hlstd{,}\hlnum{3}\hlstd{,}\hlnum{0}\hlstd{)} \hlcom{# zaokrąglenia liczb w kolumnach}
        \hlstd{)}
\end{alltt}
\end{kframe}\begin{table}

\caption{Tabela 4. Wykorzystanie wyszystkich możliwości funkcji kable + tytuł}
\begin{longtable}{llcrlr}
\toprule
  & Sepal.Length & Sepal.Width & Petal.Length & Petal.Width & Species\\
\midrule
1 & 5 & 3.5 & 1.4 & 0.2 & setosa\\
2 & 5 & 3.0 & 1.4 & 0.2 & setosa\\
3 & 5 & 3.2 & 1.3 & 0.2 & setosa\\
4 & 5 & 3.1 & 1.5 & 0.2 & setosa\\
5 & 5 & 3.6 & 1.4 & 0.2 & setosa\\
\addlinespace
6 & 5 & 3.9 & 1.7 & 0.4 & setosa\\
7 & 5 & 3.4 & 1.4 & 0.3 & setosa\\
8 & 5 & 3.4 & 1.5 & 0.2 & setosa\\
9 & 4 & 2.9 & 1.4 & 0.2 & setosa\\
10 & 5 & 3.1 & 1.5 & 0.1 & setosa\\
\addlinespace
11 & 5 & 3.7 & 1.5 & 0.2 & setosa\\
12 & 5 & 3.4 & 1.6 & 0.2 & setosa\\
13 & 5 & 3.0 & 1.4 & 0.1 & setosa\\
14 & 4 & 3.0 & 1.1 & 0.1 & setosa\\
15 & 6 & 4.0 & 1.2 & 0.2 & setosa\\
\addlinespace
16 & 6 & 4.4 & 1.5 & 0.4 & setosa\\
17 & 5 & 3.9 & 1.3 & 0.4 & setosa\\
18 & 5 & 3.5 & 1.4 & 0.3 & setosa\\
19 & 6 & 3.8 & 1.7 & 0.3 & setosa\\
20 & 5 & 3.8 & 1.5 & 0.3 & setosa\\
\addlinespace
21 & 5 & 3.4 & 1.7 & 0.2 & setosa\\
22 & 5 & 3.7 & 1.5 & 0.4 & setosa\\
23 & 5 & 3.6 & 1.0 & 0.2 & setosa\\
24 & 5 & 3.3 & 1.7 & 0.5 & setosa\\
25 & 5 & 3.4 & 1.9 & 0.2 & setosa\\
\addlinespace
26 & 5 & 3.0 & 1.6 & 0.2 & setosa\\
27 & 5 & 3.4 & 1.6 & 0.4 & setosa\\
28 & 5 & 3.5 & 1.5 & 0.2 & setosa\\
29 & 5 & 3.4 & 1.4 & 0.2 & setosa\\
30 & 5 & 3.2 & 1.6 & 0.2 & setosa\\
\addlinespace
31 & 5 & 3.1 & 1.6 & 0.2 & setosa\\
32 & 5 & 3.4 & 1.5 & 0.4 & setosa\\
33 & 5 & 4.1 & 1.5 & 0.1 & setosa\\
34 & 6 & 4.2 & 1.4 & 0.2 & setosa\\
35 & 5 & 3.1 & 1.5 & 0.2 & setosa\\
\addlinespace
36 & 5 & 3.2 & 1.2 & 0.2 & setosa\\
37 & 6 & 3.5 & 1.3 & 0.2 & setosa\\
38 & 5 & 3.6 & 1.4 & 0.1 & setosa\\
39 & 4 & 3.0 & 1.3 & 0.2 & setosa\\
40 & 5 & 3.4 & 1.5 & 0.2 & setosa\\
\addlinespace
41 & 5 & 3.5 & 1.3 & 0.3 & setosa\\
42 & 4 & 2.3 & 1.3 & 0.3 & setosa\\
43 & 4 & 3.2 & 1.3 & 0.2 & setosa\\
44 & 5 & 3.5 & 1.6 & 0.6 & setosa\\
45 & 5 & 3.8 & 1.9 & 0.4 & setosa\\
\addlinespace
46 & 5 & 3.0 & 1.4 & 0.3 & setosa\\
47 & 5 & 3.8 & 1.6 & 0.2 & setosa\\
48 & 5 & 3.2 & 1.4 & 0.2 & setosa\\
49 & 5 & 3.7 & 1.5 & 0.2 & setosa\\
50 & 5 & 3.3 & 1.4 & 0.2 & setosa\\
\addlinespace
51 & 7 & 3.2 & 4.7 & 1.4 & versicolor\\
52 & 6 & 3.2 & 4.5 & 1.5 & versicolor\\
53 & 7 & 3.1 & 4.9 & 1.5 & versicolor\\
54 & 6 & 2.3 & 4.0 & 1.3 & versicolor\\
55 & 6 & 2.8 & 4.6 & 1.5 & versicolor\\
\addlinespace
56 & 6 & 2.8 & 4.5 & 1.3 & versicolor\\
57 & 6 & 3.3 & 4.7 & 1.6 & versicolor\\
58 & 5 & 2.4 & 3.3 & 1.0 & versicolor\\
59 & 7 & 2.9 & 4.6 & 1.3 & versicolor\\
60 & 5 & 2.7 & 3.9 & 1.4 & versicolor\\
\addlinespace
61 & 5 & 2.0 & 3.5 & 1.0 & versicolor\\
62 & 6 & 3.0 & 4.2 & 1.5 & versicolor\\
63 & 6 & 2.2 & 4.0 & 1.0 & versicolor\\
64 & 6 & 2.9 & 4.7 & 1.4 & versicolor\\
65 & 6 & 2.9 & 3.6 & 1.3 & versicolor\\
\addlinespace
66 & 7 & 3.1 & 4.4 & 1.4 & versicolor\\
67 & 6 & 3.0 & 4.5 & 1.5 & versicolor\\
68 & 6 & 2.7 & 4.1 & 1.0 & versicolor\\
69 & 6 & 2.2 & 4.5 & 1.5 & versicolor\\
70 & 6 & 2.5 & 3.9 & 1.1 & versicolor\\
\addlinespace
71 & 6 & 3.2 & 4.8 & 1.8 & versicolor\\
72 & 6 & 2.8 & 4.0 & 1.3 & versicolor\\
73 & 6 & 2.5 & 4.9 & 1.5 & versicolor\\
74 & 6 & 2.8 & 4.7 & 1.2 & versicolor\\
75 & 6 & 2.9 & 4.3 & 1.3 & versicolor\\
\addlinespace
76 & 7 & 3.0 & 4.4 & 1.4 & versicolor\\
77 & 7 & 2.8 & 4.8 & 1.4 & versicolor\\
78 & 7 & 3.0 & 5.0 & 1.7 & versicolor\\
79 & 6 & 2.9 & 4.5 & 1.5 & versicolor\\
80 & 6 & 2.6 & 3.5 & 1.0 & versicolor\\
\bottomrule
\end{longtable}
\end{table}



Pytanie, czy domyślny sposób umieszczania tytułu jest dla nas satysfakcjonujący. Jeżeli nie to możemy go dopisać ręcznie zwykły (w tym przypadku wyśrodkowany) tekst, tak jak poniżej.

\begin{center}
Tabela 5. Wykorzystanie wyszystkich możliwości funkcji \textbf{kable} + tytuł
\end{center}


\begin{longtable}{llcrlr}
\toprule
  & Sepal.Length & Sepal.Width & Petal.Length & Petal.Width & Species\\
\midrule
1 & 5 & 3.5 & 1.4 & 0.2 & setosa\\
2 & 5 & 3.0 & 1.4 & 0.2 & setosa\\
3 & 5 & 3.2 & 1.3 & 0.2 & setosa\\
4 & 5 & 3.1 & 1.5 & 0.2 & setosa\\
5 & 5 & 3.6 & 1.4 & 0.2 & setosa\\
\addlinespace
6 & 5 & 3.9 & 1.7 & 0.4 & setosa\\
7 & 5 & 3.4 & 1.4 & 0.3 & setosa\\
8 & 5 & 3.4 & 1.5 & 0.2 & setosa\\
9 & 4 & 2.9 & 1.4 & 0.2 & setosa\\
10 & 5 & 3.1 & 1.5 & 0.1 & setosa\\
\bottomrule
\end{longtable}




\end{document}
